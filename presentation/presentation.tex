% !TeX encoding = UTF-8
% !TeX spellcheck = tr_TR-Turkish
\documentclass[12pt]{beamer}
\usetheme{Madrid} % Tema tercihini değiştirebilirsin
\usepackage[utf8]{inputenc}
\usepackage[T1]{fontenc}
\usepackage[turkish]{babel}
\usepackage{graphicx}
\usepackage{amsmath, amssymb, amsfonts}
\usepackage{multirow}
\usepackage{float}
\usepackage{hyperref}
\usepackage{color}

\renewcommand*\contentsname{İçindekiler}
\renewcommand{\figurename}{Şekil}
\renewcommand{\tablename}{Tablo}

\title{Talep Tahmini ve Stok Optimizasyonu ile \\ Fazla Stok \\ ve \\ Stoksuz Kalma Problemlerini Çözme}
\author{Nuh HATİPOĞLU}
\date{11 Nisan 2025}

\graphicspath{{/home/nuh/Desktop/StockAgentExample/results/}}

\begin{document}

% Title Slide
\begin{frame}
  \titlepage
\end{frame}

% Table of contents
\begin{frame}
  \frametitle{İçindekiler}
  \tableofcontents
\end{frame}

% ---------------------
\section{Giriş}
\begin{frame}
  \frametitle{Giriş}
  \begin{itemize}
    \item Hızlı moda sektöründe talep tahmini ve stok optimizasyonu önemlidir.
    \item Veri bilimi ve yapay zeka kullanarak karar destek sistemleri geliştirilmiştir.
    \item Proje, geçmiş satış verileri, ürün varyantları, kampanyalar ve kanal bilgilerini kullanmaktadır.  
    \item Amaç, talep tahminleri ve stok optimizasyon kararları vermektir.
    \item Proje üç temel fazdan oluşmaktadır:
      \begin{itemize}
        \item Sentetik veri üretimi
        \item Çoklu makine öğrenimi modeliyle talep tahmini
        \item Stok kararlarını veren bir akıllı ajan sistemi
      \end{itemize}
    \end{itemize} % Ensure \end{itemize} is properly closed
\end{frame}

\section{Sentetik Veri Üretimi ve Yapısı}
\begin{frame}
  \frametitle{Sentetik Veri Üretimi}
  \begin{itemize}
  	\item Gerçek verilere oldukça yakın şekilde 2019–2025 yılları arasını kapsayan sentetik veriler oluşturulmuştur. Bu veri seti aşağıdaki alt kümeleri içerir:
  	\begin{itemize}
  		\item products.csv: Ürün, kategori, beden ve renk varyant bilgileri
  		\item sales\_data.csv: Günlük satış adetleri (ürün, varyant, mağaza, kanal bazlı)
  		\item campaigns.csv: Kampanya tarihleri ve açıklamaları
  	\end{itemize}  

  \end{itemize}  
\end{frame}

\begin{frame}
	\frametitle{Sentetik Veri Üretimi}
	\begin{itemize}
		\item Tüm varyant kombinasyonları (örneğin Ürün A - Renk Mavi - Beden L) günlük olarak satış verisine sahiptir.
		\item Fiziksel ve Online olmak üzere iki ana kanal, 50 mağaza üzerinden modellenmiştir.
	\end{itemize}  
\end{frame}

\section{Tahmin (Forecasting) Modelleri}
\begin{frame}{Tahmin (Forecasting) Modelleri}
	Ürünlerin varyant bazlı talep tahminleri için üç farklı model kullanılmıştır: 
	\vspace{0.5cm}
	\begin{itemize}
		\item Prophet
		\item LSTM 
		\item XGBoost
	\end{itemize}
	\vspace{0.5cm}
	Her model farklı avantajlara sahiptir ve farklı veri yapılarıyla en uygun sonucu verecek şekilde tasarlanmıştır.
\end{frame}

\subsection{Prophet}
\begin{frame}{Prophet Modeli}
	\textbf{Neden Kullanıldı} \\~\\
	\begin{itemize}
		\item Facebook tarafından geliştirilen, zaman serisi verilerde trend ve mevsimsellik yakalamada güçlüdür.
		\item Kampanya etkileri gibi dışsal değişkenleri \texttt{add\_regressor} ile modele entegre edebilir.
	\end{itemize}
\end{frame}

\begin{frame}{Prophet Modeli}
	\textbf{Avantajları} 
	\begin{itemize}
		\item Model açıklanabilirliği yüksek
		\item Mevsimsel dalgalanmaları başarılı şekilde yakalar
		\item Az veriyle de çalışabilir
	\end{itemize}
	\\~\\
	\textbf{Dezavantajları} 
	\begin{itemize}
		\item Ani değişimleri (örneğin kampanya kaynaklı) yavaş öğrenir
		\item Her kombinasyon için ayrı model eğitmek gerekebilir (bu projede tek modelde çözüldü)
	\end{itemize}
\end{frame}


\subsection{LSTM}
\begin{frame}{LSTM}
	\textbf{Neden Kullanıldı} \\~\\
	\begin{itemize}
		\item Facebook tarafından geliştirilen, zaman serisi verilerde trend ve mevsimsellik yakalamada güçlüdür.
		\item Kampanya etkileri gibi dışsal değişkenleri \texttt{add\_regressor} ile modele entegre edebilir.
	\end{itemize}
\end{frame}


\subsection{XGBoost}
\begin{frame}{XGBoost}
	\textbf{Neden Kullanıldı} \\~\\
	\begin{itemize}
		\item Facebook tarafından geliştirilen, zaman serisi verilerde trend ve mevsimsellik yakalamada güçlüdür.
		\item Kampanya etkileri gibi dışsal değişkenleri \texttt{add\_regressor} ile modele entegre edebilir.
	\end{itemize}
\end{frame}

\section{Train / Test Veri Seti Hazırlanması}
\begin{frame}{Train / Test Veri Seti Hazırlanması}
	content...
\end{frame}

\section{Model Eğtim Parametreleri ve Test Sonuçları}
\begin{frame}{Model Eğtim Parametreleri ve Test Sonuçları}
	content...
\end{frame}

\subsection{Test Sonuçları}
\begin{frame}{Test Sonuçları}
	content...
\end{frame}

\section{Başarım Metrikleri}
\begin{frame}{Başarım Metrikleri}
	\begin{itemize}
		\item MAE
		\item RMSE
	\end{itemize}
\end{frame}

\section{Talep Tahmini}
\begin{frame}{Talep Tahmini}
	content...
\end{frame}


\section{Stok Optimizasyonu ve Karar Ajanı}
\begin{frame}{Stok Optimizasyonu ve Karar Ajanı}
	content...
\end{frame}

\section{Model Sonuçları}
\begin{frame}
  \frametitle{Model Ortalamaları ve Varyansları}
  \begin{tabular}{c|c|c}
    \textbf{Model} & \textbf{Ortalama Tahmin} & \textbf{Varyans} \\ \hline
    Prophet        & 9.59                     & 1.56 \\
    XGBoost        & 11.65                    & 2.63 \\
  \end{tabular}
\end{frame}

\begin{frame}
  \frametitle{EOQ, ROP, SS Karşılaştırması}
  \begin{tabular}{c|c|c|c}
    \textbf{Model} & EOQ & ROP & SS \\ \hline
    Prophet        & 380.75 & 73.90 & 6.78 \\
    XGBoost        & 419.76 & 93.01 & 11.44 \\
  \end{tabular}
\end{frame}

% ---------------------
\section{Formüller}
\begin{frame}
  \frametitle{Temel Formüller}
  \begin{equation}
    EOQ = \sqrt{\frac{2DS}{H}}
  \end{equation}
  \begin{equation}
    ROP = d \cdot L + SS
  \end{equation}
  \begin{equation}
    SS = Z \cdot \sigma_L
  \end{equation}
\end{frame}

\begin{frame}
  \frametitle{Toplam Maliyet Formülü}
  \begin{equation}
    \text{Toplam Maliyet} = \text{Sipariş Maliyeti} + \text{Stok Tutma Maliyeti} + \text{Eksik Maliyet}
  \end{equation}
\end{frame}

\begin{frame}
  \frametitle{Model Bazlı Maliyet Formülleri}
  \begin{equation}
    \text{Sipariş Maliyeti} = \frac{D}{EOQ} \cdot S
  \end{equation}
  \begin{equation}
    \text{Stok Tutma Maliyeti} = \frac{EOQ}{2} \cdot H
  \end{equation}
  \begin{equation}
    \text{Eksik Maliyet} = \frac{D}{EOQ} \cdot (ROP - EOQ) \cdot C
  \end{equation}
\end{frame}

% ---------------------
\section{Toplam Maliyetler}
\begin{frame}
  \frametitle{Toplam Maliyetler Tablosu}
  \begin{tabular}{c|c}
    \textbf{Model} & \textbf{Toplam Maliyet (TL)} \\ \hline
    Prophet        & 2039.87 \\
    XGBoost        & 2519.71 \\
  \end{tabular}
\end{frame}

\begin{frame}
  \frametitle{Sonuçlar}
  \begin{itemize}
    \item Prophet modeli daha düşük toplam maliyet sunmuştur.
    \item İşletme açısından daha karlı ve risksiz görünmektedir.
  \end{itemize}
\end{frame}

% ---------------------
\section{Risk Skorları}
\begin{frame}
  \frametitle{Varyant Bazlı Risk Analizi}
  \begin{itemize}
    \item Kırmızı, Mavi, Siyah varyantları için risk skorları: 0.855–0.884
    \item Tüm varyantlar yüksek risklidir.
    \item XGBoost daha yüksek tahmin verdiği için öncelikli tercih olabilir.
  \end{itemize}
\end{frame}

% END
\begin{frame}
  \centering
  \Huge Teşekkürler!
\end{frame}

\end{document}
