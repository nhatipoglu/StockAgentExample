% !TeX encoding = UTF-8
% !TeX spellcheck = tr_TR-Turkish
\documentclass[12pt]{beamer}
\usetheme{Madrid} % Tema tercihini değiştirebilirsin
\usepackage[utf8]{inputenc}
\usepackage[T1]{fontenc}
\usepackage[turkish]{babel}
\usepackage{graphicx}
\usepackage{amsmath, amssymb, amsfonts}
\usepackage{multirow}
\usepackage{float}
\usepackage{hyperref}
\usepackage{color}
\usepackage{booktabs} % preamble'da olmalı
\usepackage{minted}

\usepackage[most]{tcolorbox}
\definecolor{paramblue}{rgb}{0.1,0.2,0.8}

\renewcommand*\contentsname{İçindekiler}
\renewcommand{\figurename}{Şekil}
\renewcommand{\tablename}{Tablo}

\title{Talep Tahmini ve Stok Optimizasyonu ile \\ Fazla Stok \\ ve \\ Stoksuz
	Kalma Problemlerini Çözme}
\author{Nuh HATİPOĞLU}
\date{11 Nisan 2025}

\begin{document}

% Title Slide
\begin{frame}
	\titlepage
\end{frame}

% Table of contents
\begin{frame}
	\frametitle{İçindekiler}
	\begin{enumerate}
		\item Giriş
		\item Sentetik Veri Üretimi ve Yapısı
		\item Tahmin (Forecasting) Modelleri
		\item Train / Test Veri Seti Hazırlanması
		\item Model Eğitim ve Test Sonuçları
		\item Başarım Metrikleri
		\item İşletme Parametreleri ile Hesaplamalar
		\item Tahmin ve Optimizasyon
		\item Agent Analizleri
	\end{enumerate}

\end{frame}

% ---------------------
\section{Giriş}
\begin{frame}
	\frametitle{Giriş}
	\begin{itemize}
		\item Hızlı moda sektöründe talep tahmini ve stok
		      optimizasyonu önemlidir.
		\item Veri bilimi ve yapay zeka kullanarak karar destek
		      sistemleri geliştirilmiştir.
		\item Proje, geçmiş satış verileri, ürün varyantları,
		      kampanyalar ve kanal bilgilerini kullanmaktadır.
		\item Amaç, talep tahminleri ve stok optimizasyon
		      kararları vermektir.
		\item Proje üç temel fazdan oluşmaktadır:
		      \begin{itemize}
			      \item Sentetik veri üretimi
			      \item Çoklu makine öğrenimi modeliyle talep
			            tahmini
			      \item Stok kararlarını veren bir akıllı ajan
			            sistemi
		      \end{itemize}
	\end{itemize} % Ensure \end{itemize} is properly closed
\end{frame}

% ---------------------------------------------

\section{Sentetik Veri Üretimi ve Yapısı}
\begin{frame}
	\frametitle{Sentetik Veri Üretimi}
	\begin{itemize}
		\item Gerçek verilere oldukça yakın şekilde 2019–2025 yılları
		      arasını kapsayan sentetik veriler oluşturulmuştur. Bu veri seti aşağıdaki alt
		      kümeleri içerir:
		      \begin{itemize}
			      \item products.csv: Ürün, kategori, beden ve renk
			            varyant bilgileri
			      \item sales\_data.csv: Günlük satış adetleri (ürün,
			            varyant, mağaza, kanal bazlı)
			      \item campaigns.csv: Kampanya tarihleri ve açıklamaları
		      \end{itemize}

	\end{itemize}
\end{frame}

\begin{frame}
	\frametitle{Sentetik Veri Üretimi}
	\begin{itemize}
		\item Tüm varyant kombinasyonları (örneğin Ürün A - Renk Mavi -
		      Beden L) günlük olarak satış verisine sahiptir.
		\item Fiziksel ve Online olmak üzere iki ana kanal, 50 mağaza
		      üzerinden modellenmiştir.
	\end{itemize}
\end{frame}

\begin{frame}
	\frametitle{Kategori Dağılımı}
	\centering

	\includegraphics[width=0.65\linewidth]{figures/category_distribution.png}

	\vspace{0.5em}
	\small\textit{Kategori bazında ürün sayılarının dağılımı.}
\end{frame}

\begin{frame}{Sentetik Veri Üretimi}
	\begin{itemize}
		\item Ürün portföyünün büyük kısmı Erkek Giyim kategorisinde
		      yoğunlaşmıştır. Kadın ve Çocuk Giyim kategorileri daha az ürün içerir.
		      \\~\\
		\item	Bu dağılım, tahmin modellerinin veri dengesizliği
		      nedeniyle erkek ürünlerine daha duyarlı hale gelmesini sağlayabilir.
	\end{itemize}
\end{frame}

\begin{frame}
	\centering
	\includegraphics[width=0.65\linewidth]{figures/monthly_total_sales.png}
	\vspace{0.5em}
	{\small \textit{Aylık bazım toplam satış miktarları}}
\end{frame}

\begin{frame}{Sentetik Veri Üretimi}
	\begin{itemize}
		\item Haziran–Ağustos aylarında satışlar zirve yaparken, Şubat
		      ayında en düşük seviyeye gerilemektedir.
		      \\~\\
		\item Bu dönemsel etki, tahmin modellerine mevsimsellik
		      bileşenlerinin dahil edilmesini zorunlu kılar.
	\end{itemize}
\end{frame}

\begin{frame}
	\centering

	\includegraphics[width=0.65\linewidth]{figures/campaign_sales_boxplot.png}
	\vspace{0.5em}
	{\small \textit{İndirim oranı ve satış arasındaki ilişki grafiği}}
\end{frame}

\begin{frame}{Sentetik Veri Üretimi}
	\begin{itemize}
		\item Kampanya dönemlerinde uç değerlerin artması dikkat
		      çekicidir.
		      \\~\\
		\item  Ancak ortanca değerlerde belirgin fark yoktur.
		      \\~\\
		\item Bu durum, kampanyaların satışları bazı ürünlerde
		      artırdığını ama genel dağılımı çok değiştirmediğini gösterir.
	\end{itemize}
\end{frame}

\section{Tahmin (Forecasting) Modelleri}
\begin{frame}
	\frametitle{\insertsection: Tahmin (Forecasting) Modelleri}
	Ürünlerin varyant bazlı talep tahminleri için üç farklı model
	kullanılmıştır:
	\vspace{0.5cm}
	\begin{itemize}
		\item Prophet
		\item LSTM
		\item XGBoost
	\end{itemize}
	\vspace{0.5cm}
	Her model farklı avantajlara sahiptir ve farklı veri yapılarıyla en
	uygun sonucu verecek şekilde tasarlanmıştır.
\end{frame}

\begin{frame}
	\frametitle{\insertsection: Prophet Modeli}
	\textbf{Neden Kullanıldı} \\~\\
	\begin{itemize}
		\item Facebook tarafından geliştirilen, zaman serisi verilerde
		      trend ve mevsimsellik yakalamada güçlüdür.
		\item Kampanya etkileri gibi dışsal değişkenleri
		      \texttt{add\_regressor} ile modele entegre edebilir.
	\end{itemize}
\end{frame}

\begin{frame}
	\frametitle{\insertsection: Prophet Modeli}
	\textbf{Avantajları}
	\begin{itemize}
		\item Model açıklanabilirliği yüksek
		\item Mevsimsel dalgalanmaları başarılı şekilde yakalar
		\item Az veriyle de çalışabilir
		\item Veri miktarı azsa
		\item Açıklanabilirlik önemliyse
		\item Trend ve mevsimsellik barizse
		\item Operasyonel kararlarda şeffaflık gerekiyorsa
		\item Düşük varyanslı tahminler isteniyorsa
	\end{itemize}
	\\~\\
	\textbf{Dezavantajları}
	\begin{itemize}
		\item Ani değişimleri (örneğin kampanya kaynaklı) yavaş öğrenir
		\item Her kombinasyon için ayrı model eğitmek gerekebilir (bu
		      projede tek modelde çözüldü)
	\end{itemize}
\end{frame}

\begin{frame}
	\frametitle{\insertsection: LSTM}
	\textbf{Neden Kullanıldı} \\~\\
	\begin{itemize}
		\item Zaman serisi verilerde geçmiş verilerle uzun dönemli
		      bağımlılıkları modelleyebilmesi için tercih edildi.
		\item $sin/cos$ zaman kodlama, kampanya gibi ek girdilerle
		      zenginleştirildi.
	\end{itemize}
\end{frame}

\begin{frame}
	\frametitle{\insertsection: LSTM}
	\textbf{Avantajları}
	\begin{itemize}
		\item Mevsimsel yapı, trend, kampanya gibi faktörleri birlikte
		      öğrenebilir
		\item Farklı varyantları tek modelde işleyebilir
		\item Zengin zaman serisi varsa (örneğin son 3–5 yılın verisi)
		\item Karmaşık desenler ve bağımlılıklar varsa
		\item Gecikmeli/bağımlı etkiler önemliyse
		\item Öngörülen değişkenin geçmişe sıkı bağlı olduğu durumlar
	\end{itemize} \\~\\

	\textbf{Dezavantajları}
	\begin{itemize}
		\item Daha uzun eğitim süresi
		\item Hiperparametre ayarlamaları karmaşık
		\item Yorumlanabilirliği düşüktür
	\end{itemize}
\end{frame}

\begin{frame}
	\frametitle{\insertsection: XGBoost}
	\textbf{Neden Kullanıldı} \\~\\
	\begin{itemize}
		\item Facebook tarafından geliştirilen, zaman serisi verilerde
		      trend ve mevsimsellik yakalamada güçlüdür.
		\item Kampanya etkileri gibi dışsal değişkenleri
		      \texttt{add\_regressor} ile modele entegre edebilir.
	\end{itemize}
\end{frame}

\begin{frame}
	\frametitle{\insertsection: XGBoost}
	\textbf{Avantajları}
	\begin{itemize}
		\item Yüksek doğruluk
		\item Hızlı eğitim ve tahmin
		\item Öznitelik mühendisliğiyle esnek yapı
		\item Özellik mühendisliği yapılmışsa (kanal, kampanya, tarih
		      gibi)
		\item Tahmin doğruluğu ön plandaysa
		\item Daha hızlı ve agresif tahmin isteniyorsa
		\item Heterojen veri (çok değişkenli) varsa
	\end{itemize} \\~\\

	\textbf{Dezavantajları}
	\begin{itemize}
		\item Zaman bileşenleri doğrudan öğrenilmez, $sin/cos$ dönüşüm
		      gerekebilir
		\item Doğrusal olmayan yapıdan dolayı yorumlaması zordur
	\end{itemize}
\end{frame}

\begin{frame}
	\frametitle{\insertsection: Model Özelliklerinin Karşılaştırması}
	\begin{table}[h!]
		\centering

		\renewcommand{\arraystretch}{1.3}
		% Satırlar arası boşluğu artırır
		\resizebox{0.95\textwidth}{!}{
			\begin{tabular}{l l l l}
				\toprule
				\textbf{Özellik}                        & \textbf{Prophet}        &
				\textbf{XGBoost}                        & \textbf{LSTM}                                                     \\
				\midrule
				Zaman Serisi Yapısı                     & Trend + mevsimsellik
				modeller                                & Zaman bağımlılığı zayıf & Güçlü zaman bağımlılığı modeli          \\
				Açıklanabilirlik                        & Çok yüksek              & Orta                           & Düşük  \\
				Veri Miktarı İhtiyacı                   & Az                      & Orta                           & Yüksek \\
				Kampanya / dışsal etkiler               &
				\texttt{add\_regressor} ile desteklenir & Özellik olarak eklenir  & Doğrudan
				öğrenebilir                                                                                                 \\
				Eğitim Süresi                           & Çok kısa                & Kısa                           & Uzun   \\
				Genel Güvenilirlik                      & Yüksek (düşük varyans)  &
				Yüksek doğruluk                         & Karmaşık ama güçlü                                                \\
				Uç Değerlere Tepki                      & Yavaş                   & Agresif                        & Veri
				miktarına bağlı                                                                                             \\
				\bottomrule
			\end{tabular}
		}
		\caption{\small Zaman serisi modelleme yaklaşımlarının temel
			özelliklere göre karşılaştırılması}
	\end{table}
\end{frame}

\section{Train / Test Veri Seti Hazırlanması}
\begin{frame}
	\frametitle{Train / Test Veri Seti Hazırlanması}
	content...
\end{frame}

\section{Model Eğtim ve Test Sonuçları}
\begin{frame}
	\frametitle{\insertsection: Model Eğtim Parametreleri}
	content...
\end{frame}

\begin{frame}
	\frametitle{\insertsection: Test Sonuçları}
	content...
\end{frame}

\section{Başarım Metrikleri}
\begin{frame}
	\frametitle{\insertsection:  MAE –Mean Absolute Error (Ortalama Mutlak
		Hata)}
	\begin{itemize}
		\item MAE, tahmin edilen değerler ile gerçek değerler
		      arasındaki farkların mutlak değerlerinin ortalamasıdır.
		\item Hataların büyüklüğünü doğrudan ölçer.
	\end{itemize}
	\vspace{0.5cm}
	\textbf{Formül:}
	\[
		\text{MAE} = \frac{1}{n} \sum_{i=1}^{n} \left| y_i - \hat{y}_i \right|
	\]
	\vspace{0.3cm}
	\textbf{Not:} Aykırı değerlere karşı daha az hassastır.
\end{frame}

\begin{frame}
	\frametitle{\insertsection: RMSE – Root Mean Squared Error (Karekök
		Ortalama Kare Hata)}
	\begin{itemize}
		\item RMSE, tahmin hatalarının karelerinin ortalamasının
		      kareköküdür.
		\item Büyük hataları daha fazla cezalandırır.
	\end{itemize}
	\vspace{0.5cm}
	\textbf{Formül:}
	\[
		\text{RMSE} = \sqrt{ \frac{1}{n} \sum_{i=1}^{n} \left( y_i - \hat{y}_i
			\right)^2 }
	\]
	\vspace{0.3cm}
	\textbf{Not:} Aykırı değerlere karşı daha duyarlıdır.
\end{frame}

\begin{frame}
	\frametitle{\insertsection: MAE ve RMSE Karşılaştırması}
	\begin{table}
		\centering
		\small
		\begin{tabular}{l c c}
			\toprule
			\textbf{Özellik}            & \textbf{MAE}   & \textbf{RMSE} \\
			\midrule
			Hata Türü                   & Mutlak Hata    & Kare
			Hata                                                         \\
			Aykırı Değerlere Duyarlılık & Az             & Yüksek
			\\
			Yorumu                      & Daha sezgisel  & Daha
			teknik                                                       \\
			Birim                       & Orijinal birim &
			Orijinal birim                                               \\
			\bottomrule
		\end{tabular}
		\caption{\small MAE ve RMSE metriklerinin temel farklarının
			karşılaştırılması}
	\end{table}
\end{frame}

%----------------------------------------------------

\section{İşletme Parametreleri Kullanılarak Yapılan Hesaplamalar}
% EOQ
\begin{frame}
	\frametitle{Ekonomik Sipariş Miktarı (EOQ)}
	\begin{equation}
		EOQ = \sqrt{\frac{2DS}{H}}
	\end{equation}
	\begin{itemize}
		\item $D$: Yıllık talep miktarı (adet)
		\item $S$: Sipariş başına sabit maliyet
		\item $H$: Birim başına yıllık stok tutma maliyeti
	\end{itemize}

	\begin{tcolorbox}
		EOQ, toplam sipariş ve stok tutma maliyetlerini minimize edecek
		optimal sipariş miktarını belirler.
	\end{tcolorbox}

	\begin{tcolorbox}
		Stok yönetiminde maliyet etkinliğini artırır; aşırı veya
		yetersiz sipariş kaynaklı kayıpların önüne geçilmesini sağlar.
	\end{tcolorbox}
\end{frame}

% ROP
\begin{frame}
	\frametitle{Yeniden Sipariş Noktası (ROP)}
	\begin{equation}
		ROP = d \cdot L + SS
	\end{equation}
	\begin{itemize}
		\item $d$: Ortalama günlük talep
		\item $L$: Tedarik süresi (gün)
		\item $SS$: Güvenlik stoğu
	\end{itemize}

	\begin{tcolorbox}
		ROP, stok seviyesinin bu değere ulaşması durumunda yeni sipariş
		verilmesi gerektiğini gösterir.
	\end{tcolorbox}

	\begin{tcolorbox}
		Tedarik süresi boyunca stok tükenmesini önleyerek süreçlerin
		kesintisiz devam etmesini sağlar.
	\end{tcolorbox}
\end{frame}

% SS
\begin{frame}
	\frametitle{Güvenlik Stoğu (Safety Stock)}
	\begin{equation}
		SS = Z \cdot \sigma_L
	\end{equation}
	\begin{itemize}
		\item $Z$: Servis seviyesi katsayısı (örneğin \%95 için $Z
			      \approx 1.65$)
		\item $\sigma_L$: Tedarik süresindeki talep sapmasının standart
		      sapması
	\end{itemize}

	\begin{tcolorbox}
		Güvenlik stoğu, talepteki belirsizlikler veya tedarik
		gecikmeleri karşısında oluşabilecek stok yetersizliğine karşı ek stok
		miktarıdır.
	\end{tcolorbox}

	\begin{tcolorbox}
		Müşteri hizmet seviyesini artırır, stok outs (stok tükenmesi)
		riskini azaltır ve operasyonel sürekliliği destekler.
	\end{tcolorbox}
\end{frame}

\begin{frame}
	\frametitle{Toplam Maliyet Bileşenleri}
	\begin{align}
		\text{Toplam Maliyet} & = \text{Sipariş Maliyeti} \nonumber              \\
		                      & + \text{Stok Tutma Maliyeti} \nonumber           \\
		                      & + \text{Eksik Maliyet} \label{eq:toplam_maliyet}
	\end{align}

	\begin{tcolorbox}
		Toplam maliyet, stok yönetiminde dikkate alınan üç temel
		kalemden oluşur. Her bir bileşen farklı operasyonel riski temsil eder.
	\end{tcolorbox}
\end{frame}

\begin{frame}
	\frametitle{Model Bazlı Maliyet Hesaplama Formülleri}

	\begin{equation}
		\text{Sipariş Maliyeti} = \frac{D}{EOQ} \cdot S
	\end{equation}
	\begin{equation}
		\text{Stok Tutma Maliyeti} = \frac{EOQ}{2} \cdot H
	\end{equation}
	\begin{equation}
		\text{Eksik Maliyet} = \frac{D}{EOQ} \cdot (ROP - EOQ) \cdot C
	\end{equation}

	\begin{tcolorbox}
		Eksik maliyet, talep karşılanamadığında oluşan fırsat
		maliyetini temsil eder. $C$: birim başına eksiklik maliyetidir.
	\end{tcolorbox}
\end{frame}

%--------------------------------------------

\section{Tahmin ve Optimizasyon}

\begin{frame}
	\frametitle{\insertsection: Prompt Parametreleri}
	\textbf{Prompt Parametreleri}
	\begin{itemize}
		\item product\_id: 7
		\item mevcut\_stok: 22
		\item teslim\_suresi: 7
		\item siparis\_maliyeti: 50
		\item stok\_tutma\_maliyeti: 5
		\item servis\_seviyesi: 0.95
		\item stockout\_cost: 20
		\item start\_date: 2025-05-01
		\item end\_date: 2025-05-07
	\end{itemize}
\end{frame}

\begin{frame}[fragile]
	\frametitle{\insertsection: Prompt}
	\scriptsize
	\begin{verbatim}
		Ürün ID: {parameters["product_id"]}
		Tahmin edilen tarih aralığı: {parameters["start_date"]} - {parameters["end_date"]}
		İşletme Parametreleri:
		- Mevcut stok: {parameters["mevcut_stok"]}
		- Teslim süresi: {parameters["teslim_suresi"]} gün
		- Servis seviyesi: %{parameters["servis_seviyesi"] * 100}
		- Sipariş maliyeti: {parameters["siparis_maliyeti"]} TL
		- Stok tutma maliyeti: {parameters["stok_tutma_maliyeti"]} TL
		- Stoksuz kalma maliyeti (stockout cost): {parameters["stockout_cost"]} TL
		
		Prophet modeli çıktısı:
		Model Tahminleri: {df_prophet_forecast_dict}
		Stok Hesaplamaları: {stock_calculation_prophet}
		Varyant Risk Skorları: {variant_risk_prophet_df.to_dict(orient='records')}
		Toplam Maliyet Analizi: {costs_prophet}
		
		XGBoost modeli çıktısı:
		Model Tahminleri: {df_xgboost_forecast_dict}
		Stok Hesaplamaları: {stock_calculation_xgboost}
		Varyant Risk Skorları: {variant_risk_xgb_df.to_dict(orient='records')}
		Toplam Maliyet Analizi: {costs_xgboost}
	\end{verbatim}
\end{frame}

\begin{frame}[fragile]
	\frametitle{\insertsection: Prompt}
	\scriptsize
	\begin{verbatim}		
		Lütfen aşağıdaki konuları analiz et:
		
		1. Prophet ve XGBoost modellerinin genel tahmin ortalaması ve varyansı nedir?
		Hangi model daha istikrarlı ve güvenilir?
		2. Her modelin EOQ, ROP ve SS değerlerini karşılaştır. 
		Hangisi daha uygun sipariş stratejisi sunuyor?
		3. Toplam maliyet analizine göre hangi model işletmeye daha az maliyet çıkarıyor? 
		(Stok tutma, sipariş ve stoksuz kalma maliyetleri dahil)
		4. Varyant bazlı risk skorlarını incele. 
		Yüksek riskli varyantlar hangileri ve nasıl önceliklendirilmelidir?
		5. Sipariş önerisi:
		- Toplam kaç adet sipariş verilmelidir?
		- Hangi varyantlara öncelik verilmeli? (Risk skorlarına göre grupla)
		6. Tüm analizleri göz önünde bulundurarak nihai model seçimini yap.
		7. Kararını açık ve gerekçeli şekilde sun:
		- Servis seviyesi, stok-out riski, maliyetler, 
		varyant dengesi ve operasyonel uygulanabilirlik açısından değerlendir.
		- Nihai sipariş miktarını belirt ve işletmeye önerini ilet.
	\end{verbatim}
\end{frame}

\begin{frame}
	\frametitle{\insertsection: Model Bazlı Tahmin Sonuçları}
	\begin{table}
		\centering
		\small
		\begin{tabular}{l c c}
			\toprule
			\textbf{Model} & \textbf{Ortalama Tahmin} &
			\textbf{Varyans}                                 \\
			\midrule
			Prophet        & 9.59                     & 1.56 \\
			XGBoost        & 11.65                    & 2.63 \\
			\bottomrule
		\end{tabular}
		\caption{\small Prophet ve XGBoost modellerinin 7 günlük tahmin
			ortalaması ve varyans değerleri}
	\end{table}
\end{frame}

% ---------------------

\section{Agent Analizleri}
\begin{frame}
	\frametitle{\insertsection: EOQ / ROP / SS Karşılaştırması}
	\begin{table}
		\centering
		\renewcommand{\arraystretch}{1.2}
		% Satır aralığı biraz artsın
		\begin{tabular}{l|c|c|c}
			\hline \hline
			\textbf{Model} & \textbf{EOQ} & \textbf{ROP} &
			\textbf{SS}                                          \\
			\hline
			Prophet        & 380.75       & 73.90        & 6.78  \\ \hline
			XGBoost        & 419.76       & 93.01        & 11.44 \\
			\hline
		\end{tabular}
		\vspace{0.3em}
		\caption{\small EOQ, ROP ve SS değerlerinin Prophet ve XGBoost
			modellerine göre karşılaştırılması}
	\end{table}
\end{frame}

\begin{frame}
	\frametitle{\insertsection: Sipariş Stratejisi Analizi}
	\begin{itemize}
		\item XGBoost modeli daha yüksek EOQ, ROP ve SS değerlerine
		      sahiptir.
		\item  Bu, XGBoost'un daha büyük sipariş miktarları ve daha
		      yüksek güvenlik stokları önerdiği anlamına gelir.
		\item Ancak, bu durum daha fazla maliyet anlamına gelebilir.
		\item Prophet modeli daha düşük değerler sunarak daha az riskli
		      bir sipariş stratejisi sunmaktadır.
	\end{itemize}
\end{frame}

\begin{frame}
	\frametitle{\insertsection: Toplam Maliyet Karşılaştırması}
	\begin{table}
		\centering
		\small % tabloyu biraz küçültüp hizalı yapar
		\begin{tabular}{l c}
			\toprule
			\textbf{Model} & \textbf{Toplam Maliyet (TL)} \\
			\midrule
			Prophet        & 2\,039.87                    \\
			XGBoost        & 2\,519.71                    \\
			\bottomrule
		\end{tabular}
		\caption{\small Prophet ve XGBoost modellerine göre toplam stok
			maliyeti değerleri}
	\end{table}

	\textbf{Analiz: } Prophet modeli, toplam maliyet açısından daha avantajlıdır.
	Daha düşük maliyetler, işletmenin karlılığını artırır.
\end{frame}

\begin{frame}
	\frametitle{\insertsection: Varyant Bazlı Risk Skorları}
	\textbf{Yüksek Riskli Varyantlar:}

	Tüm varyantlar (Kırmızı, Mavi, Siyah) için risk skorları 0.855 ile
	0.884 arasında değişmektedir. Bu, tüm varyantların yüksek risk taşıdığını
	göstermektedir.
	\\~\\
	\textbf{Önceliklendirme:}
	Tüm varyantlar yüksek risk taşıdığı için, sipariş önceliği verilmesi
	gereken varyantlar arasında ayrım yapmak zordur. Ancak, XGBoost modelinin
	tahminleri daha yüksek olduğu için, bu modelin tahminlerine göre sipariş
	verilmesi önerilebilir.
\end{frame}

\begin{frame}
	\frametitle{\insertsection:Toplam Sipariş Miktarı ve Varyant Dağılımı}

	\begin{itemize}
		\item \textbf{Toplam Sipariş Miktarı:}
		      \begin{itemize}
			      \item Prophet: 51.90 adet
			      \item XGBoost: 71.01 adet
		      \end{itemize}

		\item \textbf{Varyantlara Öncelik:}
		      \begin{itemize}
			      \item Tüm varyantlar yüksek risk taşımaktadır.
			      \item Sipariş miktarları eşit dağıtılabilir.
			      \item XGBoost modelinin öngördüğü talep artışı dikkate
			            alınarak, varyant başına sipariş miktarı artırılabilir.
		      \end{itemize}
	\end{itemize}

\end{frame}

\begin{frame}
	\frametitle{\insertsection: Nihai Model Seçimi}

	\textbf{Seçilen Model:} \textcolor{blue!80!black}{\textbf{Prophet}}

	\vspace{0.4cm}
	Prophet modeli aşağıdaki nedenlerle tercih edilmiştir:

	\begin{itemize}
		\item Daha düşük toplam maliyetler sunar.
		\item Tahmin varyansı daha düşüktür (daha istikrarlı).
		\item Daha düşük EOQ ve ROP değerleriyle \textbf{daha az
			      sermaye bağlar}.
		\item Risk yönetimi açısından daha temkinli ve kontrollüdür.
	\end{itemize}
\end{frame}

\begin{frame}
	\frametitle{\insertsection: Nihai Karar ve İşletmeye Öneri}

	\textbf{Servis Seviyesi:} \%95 – yüksek müşteri memnuniyeti sağlar. \\
	\textbf{Stok-out Riski:} Prophet modeli ile minimize edilmiştir. \\
	\textbf{Toplam Maliyet:} Prophet modelinde daha düşüktür. \\
	\textbf{Varyant Dengesi:} Hepsi yüksek riskli, eşit dağıtım
	mantıklıdır. \\
	\textbf{Operasyonel Uygulanabilirlik:} Prophet daha basit ve
	uygulanabilir bir çözüm sunmaktadır.

	\vspace{0.5cm}
	\textbf{Nihai Sipariş Miktarı:} \textcolor{teal}{\textbf{51 adet}}
	(Her varyanta eşit dağıtım önerilir.)

	\vspace{0.4cm}
	\begin{tcolorbox}
		Prophet modeline dayanarak, toplam 51 adet sipariş verilmesi ve
		bu miktarın tüm varyantlar arasında eşit şekilde dağıtılması önerilmektedir. Bu
		yaklaşım, maliyetleri minimize ederken müşteri memnuniyetini de artıracaktır.
	\end{tcolorbox}
\end{frame}

\begin{frame}
	\frametitle{\insertsection: Nihai Karar ve İşletmeye Öneri}
	\textbf{Nihai Sipariş Miktarı:} \textcolor{teal}{\textbf{51 adet}}
	(Her varyanta eşit dağıtım önerilir.)

	\vspace{0.4cm}
	\begin{tcolorbox}
		Prophet modeline dayanarak, toplam 51 adet sipariş verilmesi ve
		bu miktarın tüm varyantlar arasında eşit şekilde dağıtılması önerilmektedir.
		\\~\\
		Bu yaklaşım, maliyetleri minimize ederken müşteri memnuniyetini
		de artıracaktır.
		\\~\\
		İşletme açısından daha karlı ve risksiz görünmektedir.
	\end{tcolorbox}

\end{frame}

\begin{frame}
	\centering
	\Huge Teşekkürler!
\end{frame}

\end{document}
